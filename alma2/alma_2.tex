\documentclass[a4paper,12pt]{scrartcl}
\usepackage[left=2.0cm,right=2.0cm,top=1.6cm,bottom=2.8cm]{geometry}
\usepackage[utf8]{inputenc}
\usepackage[T1]{fontenc}
\usepackage[ngerman]{babel}
\usepackage{amsmath}
\usepackage{amsfonts}
\usepackage{amssymb}
\usepackage{amsthm}
\usepackage{amscd}

\usepackage{fancyhdr}
\usepackage{graphicx}
\usepackage{lastpage}
\usepackage{setspace}
\usepackage{enumitem}


\onehalfspacing

\pagestyle{fancy}
\fancyhf{}
\cfoot{\thepage}

\renewcommand{\headrulewidth}{0.0pt}
\renewcommand{\footrulewidth}{0.0pt}

\setlength{\parindent}{0cm}

\newtheorem*{satz}{Satz}
\newtheorem*{lemma}{Lemma}
\newtheorem*{korollar}{Korollar}
\theoremstyle{definition}
\newtheorem*{definition}{Definition}
\newtheorem{aufgabe}{Aufgabe}[section]
\newtheorem*{beispiel}{Beispiel}
\newtheorem*{bemerkung}{Bemerkung}

\newcommand{\Id}{\operatorname{id}}

\newcommand{\Hom}{\operatorname{Hom}}
\newcommand{\End}{\operatorname{End}}
\newcommand{\Aut}{\operatorname{Aut}}
\newcommand{\Gal}{\operatorname{Gal}}
\newcommand{\GL}{\operatorname{GL}}
\newcommand{\SL}{\operatorname{SL}}
\newcommand{\PGL}{\operatorname{PGL}}


\newcommand{\Kern}{\operatorname{Kern}}
\newcommand{\Bild}{\operatorname{Bild}}

\newcommand{\Skizze}[1]{\begin{center} \includegraphics[width=8cm]{#1} \end{center}}



\newcommand{\TF}{\operatorname{TF}}
\newcommand{\CSA}{\operatorname{CSA}}
\newcommand{\Br}{\operatorname{Br}}

\newcommand{\op}{\operatorname{op}}

\newcommand{\Nrd}{\operatorname{Nrd}}
\newcommand{\N}{\operatorname{N}}


\newcommand{\Char}{\operatorname{char}}

\newcommand{\Inf}{\operatorname{Inf}}
\title{Zusammenfassung Algorithmische Mathematik II}
\begin{document}
\maketitle
\include{1.2}%Qi Cheng Hua
\subsection{Bedingte Wahrscheinlichkeiten}

\begin{definition} F\"ur Ereignisse $A$ und $B$ eines Wahrscheinlichkeitsraums
$(\Omega, \mathcal{A}, P)$ mit $P[B] \neq 0$ hei\ss t
\begin{eqnarray}
P[A|B]:=\frac{P[A \cap B]}{P[B]} \nonumber
\end{eqnarray}
die \textbf{bedingte Wahrscheinlichkeit von $A$ gegeben $B$} ("`die
Wahrscheinlichkeit daf\"ur, dass $A$ eintritt, wenn wir schon wissen, dass $B$
eintritt"').
\end{definition}

\begin{bemerkung} \quad
\begin{itemize}
\item $P[\blacktriangle|B]: A \mapsto P[A|B]$ ist eine
Wahrscheinlichkeitsverteilung auf $(\Omega, \mathcal{A})$, die \textbf{bedingte
Verteilung gegeben $B$}.
\item Der Erwartungswert $E[X|B] = \sum_{a \in S} a \cdot P[X=a|B]$ einer
diskreten Zufallsvariable $X:\Omega \rightarrow S$ bez\"uglich der bedingten
Verteilung hei\ss t \textbf{bedingte Erwartung von $X$ gegeben $B$}.
\item Im Fall der Gleichverteilung auf einer endlichen Menge gilt $P[A|B] = 
\frac{|A\cap B|}{|B|}$.
\end{itemize}
\end{bemerkung}

\subsubsection{Berechnung von Wahrscheinlichkeiten durch Fallunterscheidung}
Im Folgenden sei $\Omega = \dot{\bigcup} H_i$ eine disjunkte Zerlegung von $\Omega$
in abz\"ahlbar viele F\"alle ("`Hypothesen"').

\begin{satz} F\"ur alle $A \in \mathcal{A}$ gilt $P[A] = \sum_{i \in I, P[H_i]
\neq 0} P[A|H_i] \cdot P[H_i]$.
\end{satz}
\begin{proof} Man verwendet die $\sigma$-Additivi\"at und rechnet rum.
\end{proof}

Die Zerlegung in Hypothesen kann eventuell mehr Information als der
Gesamt\"uberblick der Situation liefern (vgl. "`Simpson-Paradoxon"' bei
Bewerbungen in Berkeley).

\subsubsection{Bayessche Regel}
Wenn man wissen will, wie wahrscheinlich die Hypothesen $H_i$ sind, kann man
zuerst $P[H_i]$ einsch\"atzen ("`a priori defree of belief"'). Wenn man dann
zus\"atzlich wei\ss , dass ein Ereignis $A \in \mathcal{A}$ mit $P[A] \neq 0$
eintritt und die bedingte Wahrscheinlichkeit $P[A|H_i]$ ("`likelyhood") f\"ur
jedes $H_i$ kennt, dann kann man eine neue Einsch\"atzung ("`a posteriori
degree of belief") erhalten, und zwar gem\"a\ss\ dem folgenden

\begin{korollar} (Bayessche Regel). F\"ur $A \in \mathcal{A}$ mit $P[A] \neq 0$
gilt
\begin{eqnarray}
P[H_i|A] = \frac{P[A|H_i] \cdot P[H_i]}{\sum_{j \in I, P[H_j] \neq 0} P[A|H_j]
\cdot P[H_j]} \nonumber
\end{eqnarray}
f\"ur alle $i \in I$ mit $P[H_i] \neq 0$, d.h. $P[H_i|A]=c \cdot P[H_i] \cdot
P[A|H_i]$, wobei $c$ eine von $i$ unabh\"angige Konstante ist.
\end{korollar}

(Man sch\"atzt $P[H_i]$ ein. Mit Hilfe von der obigen Formel rechnet man dann
aus $P[A]$ und den "`likelyhoods"' die "`neue"' Einsch\"atzung $P[H_i|A]$. Also
erh\"alt man theoretisch keine neue Information, aber diese
"`Wahrscheinlichkeiten"' sind h\"aufig nur empirische Werte und in dem Fall
kann man mit dieser Formel die bedingten Wahrscheinlichkeiten $P[H_i|A]$
einschh\"atzen.)
 % Bedingte Wahrscheinlichkeiten (Aras)
\subsection{Mehrstufige diskrete Modelle}
F\"ur ein $n$-stufiges Zufallsexperiment mit abz\"ahlbaren Stichprobenr\"aume
$\Omega_1$, $\ldots$,  $\Omega_n$ von Teilexperimenten kann $\Omega = \prod
\Omega_i$ als der Stichprobenraum des Gesamtexperiments aufgefasst werden. \\

F\"ur $\omega \in \Omega$ und $k \in \{1, \ldots, n \}$ sei $X_k(\omega) =
\omega_k$ der Ausgang des $k$-ten Experiments. Angenommen, wir kennen $P[X_1 = 
x_1] = p_1(x_1)$ f\"ur alle $x_1 \in \Omega_1$ und $P[X_k=x_k|X_1=x_1, \ldots,
X_{k-1}=x_{k-1}] = p_k(x_k|x_1, \ldots, x_{k-1})$ f\"ur alle $k \in \{ 1,
\ldots, n \}$. Dann k\"onnen wir die gesamte Wahrscheinlichkeitsverteilung
$P$ auf $\Omega$ folgenderma\ss en erhalten:

\begin{satz} Seien $p_1$ und $p_k(\blacktriangle|x_1, \ldots, x_{k+1})$ f\"ur
jedes $k=2, \ldots, n$ und $x_i \in \Omega_i$ die Massenfunktion einer
Wahrscheinlichkeitsverteilung auf $\Omega_k$. Dann existiert genau eine
Wahrscheinlichkeitsverteilung $P$ auf $(\Omega, \mathcal{P}(\Omega ))$, die die
obige Eigenschaften hat. Diese ist bestimmt durch die Massenfunktion
\begin{center}
$p(x_1, \ldots, x_n) = p_1(x_1) p_2(x_2 | x_1) p_3(x_3 | x_1, x_2) \cdots p_n(
x_n | x_1, \ldots, x_{n-1})$.
\end{center}
\end{satz}
\begin{proof} Rumrechnerei. Die Eindeutigkeit folgt aus der Existenz.
\end{proof}

\subsubsection{Produktmodelle}
Ist der Ausgang des $i$-ten Experiments unabh\"angig von $x_1,\ldots ,x_{i-1}$,
so gilt $p_i(x_i|x_1, \ldots, x_{i-1}) = p_i(x_i)$ mit einer von $x_1,\ldots ,
x_{i-1}$ unabh\"angigen Massenfunktion $p_i$ einer
Wahrscheinlichkeitsverteilung $P_i$ auf $\Omega_i$. In diesem Fall gilt
$p(x_1, \ldots, x_n) = \prod_{i=1}^n p_i(x_i)$ f\"ur alle $(x_1, \ldots, x_n)
\Omega$.

\begin{definition} Die Wahrscheinlichkeitsverteilung $P$ auf $\Omega = \prod
\Omega_i$ mit obiger Massenfunktion hei\ss t \textbf{Produkt} von $P_1$,
$\ldots$, $P_n$ und wird mit $P_1 \otimes \ldots \otimes P_n$ notiert.
\end{definition}

\begin{beispiel} Bernouilliverteilung.
\end{beispiel}

\begin{satz} Im Produktmodell gilt f\"ur beliebige Ereignisse $A_i \subseteq
\Omega_i$
\begin{eqnarray}
P[A_1 \times \ldots \ A_n] = P[X_1 \in A_1, \ldots, X_n \in A_n] =
\prod_{i=1}^n P[X_i \in A_i] = \prod_{i=1}^n P_i[A_i], \nonumber
\end{eqnarray}
d.h. $X_1, \ldots, X_n$ sind unabh\"angige Zufallsvariablen (s. n\"achstes
Kapitel).
\end{satz}
\begin{proof} Rechnung.
\end{proof}

\subsubsection{Markov-Ketten}
Wir wollen eine zufa\"allige Entwicklung mit abz\"ahlbarem Zustandsraum $S$
modellieren. Dazu betrachten wir den Stichprobenraum $\Omega = S^{n+1}$.
H\"aufig h\"angt die Weiterentwicklung des Systems nur vom gegenw\"artigen
Zustand ab, d.h. es gilt $p_k(x_k| x_0, \ldots, x_{k-1}) = p_k(x_{k-1}, x_k)$
("`Bewegungsgesetz"'), wobei f\"ur $p_k: S \times S \rightarrow [0,1]$
\begin{enumerate}
\item $p_k(x,y) \geq 0$ f\"ur alle $x,y \in S$ und
\item $\sum_{y \in S} p_k(x,y) = 1$
\end{enumerate}
gelten. (Dies bedeutet, dass $p_k(x, \blacktriangle)$ f\"ur jedes $x \in S$ die
Massenfunktion einer Wahrscheinlichkeitsverteilung auf $S$ ist.)

\begin{definition} Eine "`Matrix"' $p_k(x,y)$ mit den obigen Eigenschaften
hei\ss t \textbf{stochastische Matrix} (oder \textbf{stochastischer Kern}) auf
$S$. \\
(Im Mehrstufenmodell gilt in dieser Situation $p(x_0, \ldots, x_n) = p_0(x_0)
p_1(x_1,x_2) \cdots p_n(x_{n-1}, x_n)$.) \\
Der Fall, in dem $p_k(x,y) = p(x,y)$ unabh\"angig von $k$ ist, nennt man
\textbf{zeitlich homogen}.
\end{definition}

\begin{beispiel} \quad
\begin{itemize}
\item Produktmodell
\item Random Walk auf $\mathbb{Z}^d$
\item Urnenmodelle
\end{itemize}
\end{beispiel}

\subsubsection{Berechnung von Wahrscheinlichkeiten}

\begin{satz} (Markov-Eigenschaft) F\"ur alle $0 \leq k < l \leq n$ und $x_0,
\ldots, x_l \in S$ mit $P[X_0 = x_0, \ldots , X_k = x_k ] \neq 0$ gilt
\begin{eqnarray}
P[X_l = x_l | X_0=x_0, \ldots, X_k] = P[X_l = x_l | X_k = x_k] =
(p_{k+1}p_{k+2} \cdots p_l)(x_k,x_l), \nonumber
\end{eqnarray}
wobei $(pq)(x,y):=\sum_{z\in S} p(x,z) q(z,y)$ das Produkt der Matrizen $p$ und
$q$ ist.
\end{satz}
\begin{proof} Indexschlacht und Rechnungskampf.
\end{proof} % Mehrstufige diskrete Modelle (Aras)
% Abschnitt 2.3 Unabhaengigkeit von Ereignissen (Leon)
\include{unabhaengigkeit}
\include{2.4}%Qi Cheng Hua
\end{document}
