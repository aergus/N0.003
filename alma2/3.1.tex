\subsection{Varianz und Kovarianz}

Sei $(\Omega,\mathcal{A},P)$ ein Wahrscheinlichkeitsraum und $X:\Omega\rightarrow S$
eine Zufallsvariable auf $(\Omega,\mathcal{A},P)$, so dass   $E[|X|]$
endlich ist.

\begin{definition}

\[
Var(X):=E[(X-E[X])^{2}]
\]


hei\ss t $\mathbf{Varianz}$ von $X$ und liegt in $[0,\infty]$.

\[
\sigma(X):=Var(X)^{1/2}
\]


hei\ss t $\mathbf{Standardabweichung}$von $X$.

\end{definition}

Interpretation: Kennzahl f\"ur die Gr\"o\ss e der Fluktuationen von $X$
um $E[X]$; Ma\ss  f\"ur Risiko bei Prognose des Ausgangs $X(\omega)$
durch $E[X]$.

\begin{bemerkung}
\begin{itemize}
\item $Var(X)=\sum_{a\in S}(a-m)^{2}p_{X}(a)$, wobei $m=E[X]=\sum_{a\in S}a\sum_{a}p_{X}(a)$.
\item $Var(X)=0$ gdw. $P[X=E[X]]=1${]}.
\item $Var(X)=E[X^{2}]-E[X]^{2}$.
\item $Var(aX+b)=Var(aX)=a^{2}Var(X)$.
\end{itemize}
\end{bemerkung}
\begin{beispiel}
\begin{itemize}
\item Sei $X=1$ mit Wahrscheinlichkeit $p$ und $X=0$ mit Wahrscheinlichkeit
$1-p$. Dann ist $Var(X)=p(1-p)$.
\item Sei $T$ geometrisch verteilt mit Parameter $p\in(0,1].$ Dann ist
$Var(T)=\frac{1-p}{p^{2}}$.
\end{itemize}
\end{beispiel}
\begin{definition}

\[
\mathcal{L}^{2}(\Omega,\mathcal{A},P):=\{X:\Omega\rightarrow\mathbb{R}|E[X^{2}]<\infty\}
\]

\end{definition}
\begin{lemma} 
\begin{itemize}
\item F\"ur Zufallsvariablen $X,Y\in\mathcal{L}^{2}$gilt: $E[|XY|]\leq E[X^{2}]^{1/2}E[Y^{2}]^{1/2}<\infty$
\item $\mathcal{L}^{2}$ ist ein Vektorraum und $(X,Y)_{\mathcal{L}^{2}}:=E[XY]$
ist eine positiv semidefinite symmetrische Bilinearform (Skalarprodukt)
auf $\mathcal{L}^{2}$. Insbesondere gilt die Cauchy-Schwarz-Ungleichung.
\item F\"ur $X\in\mathcal{L}^{2}$ gilt $E[|X|]<\infty$
\end{itemize}
\end{lemma}
\begin{definition}

Seien $X,Y\in\mathcal{L}^{2}$.
\begin{itemize}
\item $Cov(X,Y):=E[(X-E[X])(Y-E[X])]=E[XY]-E[X]E[Y]$ hei\ss t $\mathbf{Kovarianz}$
von X und Y.
\item Gilt $\sigma(X),\sigma(Y)\neq0$, so hei\ss t $\varrho(X,Y):=\frac{Cov(X,Y)}{\sigma(X)\sigma(Y)}$
$\mathbf{Korrelationskoeffizient}$von X und Y.
\item X und Y hei\ss en $\mathbf{unkorreliert}$, falls $Cov(X,Y)=0$, d.h
falls $E[XY]=E[X]\cdot E[Y]$.
\end{itemize}
\end{definition}
\begin{satz}

Seien $X:\Omega\rightarrow S$ und $Y:\Omega\rightarrow T$ diskrete
Zufallsvariablen auf $(\Omega,\mathcal{A},P)$. Dann sind \"aquivalent:
\begin{itemize}
\item X und Y sind unabh\"angig
\item f(X) und g(Y) sind unkorreliert f\"ur alle Funktionen $f:S\rightarrow\mathbb{R}$
und $g:T\rightarrow\mathbb{R}$ mit $f(X),g(Y)\in\mathcal{L}^{2}$.
\end{itemize}
\end{satz}
\begin{beispiel} 
Sei $X=1,0,-1$, jeweils mit Wahrscheinlichkeit $\frac{1}{3}$,
und $Y=X^{2}$. Dann sind $X$ und $Y$ nicht unabh\"angig, aber unkorreliert.
Intuition: Unkorelliertheit bedeutet nur kein linearer Zusammenhhang.
Hier liegt ein quadratischer Zusammenhang vor.
\end{beispiel}
\begin{satz}

F\"ur $X_{1},...,X_{n}\in\mathcal{L}^{2}$ gilt:

\[
Var(X_{1}+...+X_{n})=\sum_{i=1}^{n}Var(X_{i})+2\sum_{i,j=1,i<j}^{n}Cov(X_{i},X_{j})
\]
\end{satz}
